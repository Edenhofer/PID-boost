\chapter*{Abstract}

This study aims at evaluating particle identification approaches.

At first the goodness of the detector yield is measured. Flaws are revealed and possible causes evaluated. In addition, current techniques for combining detector variables are outlined.

Next, a Bayesian approach to particle identification is discussed. It aims at producing probabilities of a track belonging to a particle species in dependance of the received signals. The process of obtaining conditional probabilities is described in detail. Furthermore some extensions to the Bayesian approach are presented and evaluated. Flaws and benefits are compared using a generic decay.

Finally, a neural network is used to label particle tracks. Different methods to adapt the weights and various pre-processing steps are evaluated for a simple network. Hereby, tools from machine learning and statistics are discussed and their application is outlined. Last but not least, the accuracy of the network on a generic decay is determined and a comparison with the Bayesian approaches is performed.
