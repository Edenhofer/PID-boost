\chapter*{Abstract}

This study aims at evaluating further particle analysis techniques.
At first the goodness of the detector variables is measured. Flaws are outlined and possible causes are evaluated. Afterwards further techniques are discussed which combine the detector variables in a new way.

A Bayesian approach to particle identification is discussed. It aims at producing probabilities of a track belonging to a particle species in dependance of the received signal. The process of obtaining the conditional probabilities is described in greater detail. In addition, some extensions to the Bayesian approach are presented and evaluated. Flaws and benefits are compared using a generic decay.

Lastly, a neural network is used to label particle tracks. For a simple network, different methods to adapt the weights and various input forms are evaluated. Hereby, tools from machine learning and statistics are discussed and their application is outlined. In the end the accuracy of the network on a generic decay is determined.
