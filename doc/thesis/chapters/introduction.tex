\chapter{Introduction}
\label{chap:introduction}

Since the dawn of physics one of the most fundamental question has been to find the most elemental constituents of matter. In this regard, the standard model has proven to be extremely useful. It postulates six quark species, three charged kinds of leptons, a neutral neutrino for every charged lepton species and four species of gauge bosons as well as the Higgs boson. Besides its achievement in describing the very basic principals of interactions and its experimental predictions being very precise, there emerged significant flaws; namely the high energy limit, matter \& anti-matter asymmetry and the existence of dark matter and dark energy.

The Belle~\RN{2} experiment was designed to search for new physics phenomena with a massive volume of data in the flavour sector. The violation of $CP$-symmetry had already been observed at the predecessor experiment Belle. However, analysis of rare decays with only small corrections to the Standard Model are hard to spot and require more statistics. At the hardware frontier, the detector and accelerator system were updated. The software side required adaption as well to cope with the massive amount of data.

In order to spot such small deviations in such large sets of data, a reliable particle identification is a fundamental requirement. Its role is to assign particle species labels to tracks which are identified by the detector system. By doing so, it allows further analysis of events to better focus on the particles relevant to them.
