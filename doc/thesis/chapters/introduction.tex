\chapter{Introduction}
\label{chap:introduction}

One of the most fundamental challenges in physics is to find the elemental constituents of matter. In this regard, the Standard Model has proven to be extremely useful. It postulates six quark species, three charged kinds of leptons, a neutral neutrino for every charged lepton species and four species of gauge bosons in addition to the Higgs boson. Besides its achievement in describing the very basic principals of interactions and its experimental predictions being very precise, there are significant flaws; namely the lack of a description of gravitation, matter \& anti-matter asymmetry and the existence of dark matter and dark energy.

The violation of $CP$-symmetry in $B$ mesons had already been observed at the predecessor experiment Belle. The upgrade of the experiment, named Belle~\RN{2} is designed to search for new physics phenomena with a massive volume of data in the flavour sector. The volume is important as rare decays with only small corrections to the Standard Model are difficult to analyse properly and require high statistics. At the hardware frontier, the detector and accelerator system were updated. In addition, the software side required adaption to cope with the massive amount of data.

In order to recognize such small deviations in such large sets of data, a reliable particle identification is an essential requirement. Its role is to assign particle species labels to tracks which are identified by the detector system. By doing so, it allows further analysis of events to better focus on the particles relevant to them.
