\chapter{Belle~\RN{2}}
\label{chap:belle2_experiment}

\section{Experiment}
\label{sec:experimental}

The Belle~\RN{2} experiment is performed at the SuperKEKB accelerator located in Tsukuba, Ibaraki Prefecture, Japan. It is designed to study the B meson and the violation of the CP symmetry.
At the experiment asymmetric electron-, positron-beams are collided with a center of mass energy of $\sqrt{S} = 10.58 \mathrm{~GeV}$, just right to create the $\Upsilon (4S)$ meson. The two beams, positrons at $4 \mathrm{~GEV}$, electrons at $7 \mathrm{~GEV}$ are focussed to a cross section of $~10 \mathrm{\mu m} \times 60 \mathrm{nm}$. The additional boost in one direction is used to get a time measurement of the B-lifetime. %TODO

In comparison to the predecessor experiment Belle the integrated luminosity will be $50$ times higher at $50~{ab}^{-1}$ and the instantaneous luminosity has been increased $40$-fold to $8 \cdot 10^{35} \mathrm{cm}^{-2} \mathrm{s}^{-1}$.

\section{Detector system}
\label{sec:detector_system}

\subsection{Overview}
\label{subsec:detector_system_overview}

If not specifically stated otherwise the charge conjugate of a particle shall be assumed to be included if a particle is mentioned.

The Belle~\RN{2} detector system is a composition of multiple detectors, each measuring a subset of properties of particle's track.
The inner three detectors --- \textbf{P}i\textbf{X}el \textbf{D}etector (PXD), \textbf{S}ilicon \textbf{V}ertex \textbf{D}etector (SVD) and \textbf{C}entral \textbf{D}rift \textbf{C}hamber (CDC) --- record the position of particles. Hence they are also called tracking detectors. They are located in a homogeneous magnetic field of $1.5~\mathrm{T}$.
The innermost detector is the PXD. Together with the SVD, located right after it, they are used to reconstruct decay vertices and identify tracks belonging to particles with a low-momentum.
The CDC measures the momentum and charge of particles via their curvature in the magnetic field.
Next the \textbf{T}ime \textbf{O}f \textbf{P}ropagation counter (TOP) (`Barrel PID' in figure~\ref{fig:belle2_detector_design_white_paper}) and the \textbf{A}erogel \textbf{RICH} counter (ARICH) (`Endcap PID' in figure~\ref{fig:belle2_detector_design_white_paper}) are used to identify charged particles via their emission of Cherenkov radiation in the detector. However there is no such installation for the backwards-facing endcap of the detector.
The \textbf{E}lectromagnetic \textbf{C}a\textbf{L}orimeter (ECL) identifies photons and electron.
The outermost detector called $\boldsymbol{K}^0_{\boldsymbol{L}}$/$\boldsymbol{\mu}$ (KLM) is used to identify kaons and muons.

\begin{figure}[ht]
    \centering
    \includegraphics[width=\textwidth,height=0.6\textheight,keepaspectratio]{{{../res/Belle 2 detector design white paper (truncated)}}}
    \caption{Side view of the upper half of the Belle~\RN{2} detector. The bottom half (not shown in the picture) would correspond to the mirror image alongside the beampipe. Adapted from~\cite{Abe:2010gxa}.}
    \label{fig:belle2_detector_design_white_paper}
\end{figure}

\subsection{Silicon detectors}
\label{subsec:detector_system_silicon_detectors}

The PXD and SVD consist of tiny doped silicon chips which yield the location of electron-holes created by particles passing through it. The PXD detector uses small pixels while the SVD detector uses strips of detector materials. Therefore the PXD detector is able to further differentiate multiple simultaneous tracks while the SVD allows for a faster readout and is less prone to noise.

\subsection{Central drift chamber}
\label{subsec:detector_system_tracking_detectors}

The CDC, which surrounds the PXD and SVD, consists of a collection of charged wires and detection wires located in a gas filled volume. The detection wires are used to measure the drop in voltage produced by electromagnetic showers created by particles passing through the gas. The wires are close to being parallel to the beampipe but have a slight twist. This allows the detector to not only have an excellence estimation of the transverse distance to the beam pipe but also provides information about the longitudinal position.

\subsection{Barrel and endcap PID}
\label{subsec:detector_system_barrel_and_endcap_pid}

In the TOP and ARICH detector the Cherenkov effect is used to measure the momentum and velocity of particles. Charged particles which travel faster than the speed of light in the medium --- Quartz in case of the TOP detector and aerogel for the ARICH detector --- produce light due to their uneven polarization of the medium. By measuring the time of propagation and the angle of the emitted light (Cherenkov angle) an estimation of the velocity as well as the momentum can be calculated.

\subsection{Electromagnetic calorimeter}
\label{subsec:detector_system_electromagnetic_calorimeter}

The main purpose of the ECL detector is to measure the position and energy of photons and electrons. Both particle species excite the medium and create electromagnetic showers. The light from the de-excitation can subsequently be measured.

\subsection{$\boldsymbol{K}^0_{\boldsymbol{L}}$/$\boldsymbol{\mu}$ detector}
\label{subsec:detector_system_k0lmu}

Last but not least the KLM detector measures the penetration ability of particles in order to differentiate between muons and kaons. When flying through the detector the particles passes through plates serving as electrodes separated by a layers of gas in between them. Created ionized particles are accelerated in this field and subsequently produce a spark picked up by the detector via measuring the voltage.

\section{Interaction with matter}
\label{sec:interaction_with_matter}

\subsection{Charged particle interaction}
\label{subsec:interaction_with_matter}

Particles with a non-zero charge mainly interact with the medium electromagnetically. An interaction occurs either by scattering in the electric field of the atom, polarization of the medium, ionization, excitation or scattering at the atom itself. Positrons and electrons additionally have the ability to annihilate.

The polarization of the medium causes Cherenkov radiation to be emitted. At velocities below the speed of light in the medium ($v < c/n$) nothing of interest may be observed. However at $v > c/n$ the Cherenkov effect must be taken into account. The effect occurs due to the information about the charge of the traversing particle not or reaching the medium in front of it soon enough. Hence medium behind the particle being already already aligned with the electric field while the medium in front not being it. This results in an electromagnetic wave. The angle between the normal vector of the wave and the track of the particle is given by
\begin{equation*}
    \cos(\Theta_{c}) = \frac{1}{v/c \cdot n} = \frac{1}{\beta \cdot n}
    \mathrm{.}
\end{equation*}
It is the effect on which the TOP and ARICH detectors build upon.

Particles with a low-energy traveling through a medium interact predominantly with atomic electrons. The average energy loss of a particle is described by the Bethe-Bloch formula. For velocities of up to about $90\%$ of the speed of light it is given by $<\mathrm{d}E/\mathrm{d}x> \propto 1/{\beta^2}$ and for $\beta \gamma \approx 4$ it is minimal. As such it describes the momentum dependency of the average energy loss. For different particle species it gets shifted horizontally. However the actual $mathrm{d}E/\mathrm{d}x$ distribution is modelled by the Landau distribution. Note that the assumptions needed for this ansatz are not met for electrons as both particles of the interaction belong to the same group and their mass is too similar.
A $\mathrm{d}E/\mathrm{d}x$-measurement is performed at the silicon detectors and the CDC.

For high-energy particles the loss in energy by interacting with the electromagnetic field of the nucleus is the dominant part. Energy is radiated away via so called Bremsstrahlung. The left over energy decreases exponentially with the distance traversed and is inversely proportional to the square root of the mass. Therefore it is most important for particles with a low mass, e.g. electrons.
The radiation due to this effect is mainly measured at the ECL.

\subsection{Particle identification}
\label{subsec:particle_identification}

One of the most essential measurement for classifying particles below $1 \mathrm{~GeV/c}$ is the $\mathrm{d}E/\mathrm{d}x$-measurement from the silicon detectors and the CDC.
