\chapter{Computer code}
\label{chap:computer_code}

The repository for the programs and the thesis is hosted at \href{https://github.com/Edenhofer/PID-boost}{GitHub:Edenhofer/PID-boost}. It includes the slides for the weekly updates, all required graphics for building the PDFs, and some documentation.

The core of the program is divided into three main parts: the program for generating visuals~\lstinline|event_metrics.py|, the program~\lstinline|event_nn.py| used for training the neural network and the all encompassing library~\lstinline|lib.py|. Further scripts are rather short compared to those three ones. The smaller programs were used, e.g., to simulate a decay, generate decay data sets or to visualize parameters of a neural network. All important features are documented via an entry in the program's \lstinline|help|~page. Functions in the code itself which are deemed not self-explanatory are described via a Python~\lstinline|docstring|.

Probably of most interest is the program~\lstinline|event_metrics.py|. It was used for all plots which compare two approaches. Its parameters are further divided into groups called \lstinline|actions|, \lstinline|sub-options| and \lstinline|utility-options|. An \lstinline|action| induces some form of calculation or visualization which can be further parameterized via \lstinline|sub-options|. The \lstinline|utility-options| group allows for controlling the input output behavior of the script.

For example, the parameter \lstinline|--multivariate-bayes| would be an action. It can be further configured by, e.g., specifying the number of cuts to perform on the output of the classification functions via \lstinline|--ncuts|. The number of bins for the variables on which the a priori probability may depend on, can be set via \lstinline|--nbins|. By using \lstinline|-i| and \lstinline|-o|, one may configure the input and the output location.

Several of the previously described parameters are valid for other scripts in the repository as well. Sensible default parameters alongside multiple examples can be found in the readme file of the repository.
