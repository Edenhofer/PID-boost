\chapter{Conclusion}
\label{chap:conclusion}

The particle identification at Belle~\RN{2} is described in~\autoref{sec:pid_variables}. It gives a brief overview of the particle samples used for further analysis and describes the current and future particle identification variables. Furthermore the chapter discusses the goodness of the global PID variables. Overall the likelihood ratios are properly defined however flaws in the definition of the global PID of the deuteron are revealed. The unexpected kink can be traced back to the CDC detector and was especially dominant for particles with a low to medium transverse momentum.

\autoref{sec:bayesian_approach}~discusses an extension to the particle identification based on the Bayesian theorem. The likelihood of different particles are weighted by specie and are assigned a priori probabilities. Those a priori probabilities at first only depend on the overall abundance of the specie and showed similar results to the global PID approach with an overall improvement in the identification efficiency. The introduction of further dependencies of the a priori probabilities increases the overall performance only marginally.

Lastly an alternative approach in identifying particles using a neural network is analysed in~\autoref{sec:neural_network_approach}. Using a simple network provides a descent performance with overall accuracies in the range of the previously discussed Bayesian approach. Nevertheless, it is heavily biased by particles of high abundance and upsampling only shows limited improvements.

Overall the particle identification can be slightly improved by using a Bayesian approaches. Particle identification rates for specific specie depend on the identification algorithm and may be chosen depending on the decay of interest. The neural network does not provide a sufficiently good classification to be used right away but is an especially promising candidate for further research.
