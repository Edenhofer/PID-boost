\chapter{Conclusion}
\label{chap:conclusion}

The particle identification at Belle~\RN{2} is described in~\autoref{sec:pid_variables}. It gives a brief overview of the particle samples used for further analysis and describes the current and future particle identification variables. Furthermore, the chapter discusses the goodness of the global PID variables. Overall, the likelihood ratios are properly defined, however, flaws in the definition of the global PID of the deuteron are revealed. The unexpected kink can be traced back to the CDC detector and is especially dominant for particles with a low to medium transverse momentum.

In~\autoref{sec:bayesian_approach} an extension of the particle identification based on the Bayesian theorem is discussed. The likelihood of different particles are weighted by species and are assigned a priori probabilities. Those a priori probabilities at first only depend on the abundance of a species and show similar results to the global PID approach. The new approach increases the overall identification efficiencies by increasing the weights of particles with a high abundance. The introduction of further dependencies of the a priori probabilities increases the performance only marginally.

Finally, an alternative approach in identifying particles using a neural network is analysed in~\autoref{sec:neural_network_approach}. Using a simple network provides a descent performance with overall accuracies in the range of the previously discussed Bayesian approach. Nevertheless, it is heavily biased by particles of high abundance and upsampling the data only shows limited improvements.

Last but not least, the particle identification can be improved by using a Bayesian approach. Particle identification rates for specific species highly depend on the identification algorithm and may be chosen depending on the decay of interest. The neural network does not provide a sufficiently good classification to be used right away but is an especially promising candidate for further research.
