\chapter{Conclusion}
\label{chap:conclusion}

Three main approaches for the particle identification at Belle~\RN{2} are studied.

First, the current and future particle identification variables are described. Hereby, the goodness of the global PID variables is discussed. Altogether, the likelihood ratios are properly defined, however, a flaw in the definition of the global PID of the proton is revealed. The unexpected kink can be traced back to the CDC detector and is especially dominant for particles with a low to medium transverse momentum.

On the basis of the particle likelihoods, further extensions of the particle identification utilizing Bayes theorem are discussed. The likelihood of different particles are weighted by species and are assigned a priori probabilities. At first, these a priori probabilities only depend on the abundance of a species and show similar results to the global PID approach. The new approach increases the overall identification efficiencies by increasing the weights of particles with a high abundance. The introduction of further dependencies of the a priori probabilities increases the performance only marginally.

Finally, an alternative approach in identifying particles using a neural network is analysed. Using a simple network provides acceptable performance with overall accuracies in the range of the previously discussed global PID. Nevertheless, it is heavily biased by particles of high abundance and upsampling the data shows only limited improvements.

In conclusion, the particle identification can be improved by using a Bayesian approach. However, particle identification rates for specific species highly depend on the identification algorithm and may be chosen depending on the decay of interest. Currently, the neural network does not provide a sufficiently good classification to be used right away, but is an especially promising candidate for further research.
