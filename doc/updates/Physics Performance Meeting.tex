\documentclass{beamer}
	\usepackage[utf8]{inputenc}		% required for umlauts
	\usepackage[english]{babel}		% language
	%\usepackage[sfdefault]{roboto}	% enable sans serif font roboto
	%\usepackage{libertine}			% enable this on Windows to allow for microtype
	\usepackage[T1]{fontenc}		% required for output of umlauts in PDF

	\usepackage{mathtools}		% required for formulas

	\usepackage{caption}		% Customize caption aesthetics
	\usepackage{tcolorbox}		% fancy colored boxes
	\usepackage{xcolor}			% Highlighting
	\usepackage{soul}
	\usepackage{graphicx}		% required to insert images
	\usepackage[space]{grffile} % insert images baring a filename which contains spaces
	\usepackage{float}			% allow to forcefully set the location of an object

	\usepackage[tracking=true]{microtype} % required to change character spacing

	\usepackage[style=numeric,backend=biber]{biblatex}
	\usepackage{hyperref}		% insert clickable references

	\usepackage{datetime}		% flexible date specification
	\newcommand{\leadingzero}[1]{\ifnum#1<10 0\the#1\else\the#1\fi}
	\newcommand{\todayddmmyyyy}{\leadingzero{\day}.\leadingzero{\month}.\the\year}
	\newcommand{\mathcolorbox}[2]{\colorbox{#1}{$\displaystyle #2$}}

	\usepackage{geometry}
	\usepackage{scrextend}		% allow arbitrary indentation

	\usepackage{color}

	\setbeamercolor{title}{fg=orange}
	\setbeamertemplate{title}{
		\color{orange}
		\textbf{\inserttitle}
	}
	\setbeamercolor{tableofcontents}{fg=orange}
	\setbeamercolor{section in toc}{fg=black}
	\setbeamercolor{subsection in toc}{fg=black}
	\setbeamertemplate{frametitle}{
		%\vspace{0.5em}
		\color{orange}
		\begin{center}
			\textbf{\insertframetitle} \\
			{\small \insertframesubtitle}
		\end{center}
	}
	\setbeamertemplate{itemize item}{\color{black}$\bullet$}
	\setbeamertemplate{itemize subitem}{\color{black}$\circ$}
	\setbeamercolor{block title}{fg=black}
	\captionsetup{font=scriptsize,labelfont={bf,scriptsize}}

	\title{Bachelor Thesis on `Optimization~of~Particle~Identification'}
	\subtitle{Preliminary Studies}
	\author[Edenhofer]{Gordian Edenhofer}
	\institute[LMU]{
		Working Group of Prof.~Kuhr, Faculty of Physics \\
		University of Munich
	}
	\date{Physics Performance Meeting, 29. Mai 2018}
	\subject{Particle Physics}


\begin{document}

\section{Neyman Pearson}
\subsection{Intuition}
\begin{frame}
	\frametitle{\insertsection}
	\framesubtitle{\insertsubsection}

	\begin{figure}
		\centering
		\includegraphics[width=\textwidth,height=0.7\textheight,keepaspectratio]{{{../res/Neyman-Pearson Visualization}}}
		\caption{Example for relative pi Abundance over Likelihood Ratio}
	\end{figure}
\end{frame}

\subsection{`ALL' detector; charge insensitive}
\begin{frame}
	\frametitle{\insertsection}
	\framesubtitle{\insertsubsection}

	\begin{figure}
		\centering
		\includegraphics[width=\textwidth,height=0.7\textheight,keepaspectratio]{{{../res/charged 01/General Purpose Statistics: Relative pi Abundance in Likelihood Ratio Bins for ALL detector}}}
		\caption{Relative pi Abundance in Likelihood Ratio Bins for `ALL' detector}
	\end{figure}
\end{frame}

\begin{frame}
	\frametitle{\insertsection}
	\framesubtitle{\insertsubsection}

	\begin{figure}
		\centering
		\includegraphics[width=\textwidth,height=0.7\textheight,keepaspectratio]{{{../res/charged 01/General Purpose Statistics: Relative K Abundance in Likelihood Ratio Bins for ALL detector}}}
		\caption{Relative K Abundance in Likelihood Ratio Bins for `ALL' detector}
	\end{figure}
\end{frame}

\subsection{`ALL' detector; negatively charged}
\begin{frame}
	\frametitle{\insertsection}
	\framesubtitle{\insertsubsection}

	\begin{figure}
		\centering
		\includegraphics[width=\textwidth,height=0.7\textheight,keepaspectratio]{{{../res/charged 01/General Purpose Statistics: Relative pi- Abundance in Likelihood Ratio Bins for ALL detector}}}
		\caption{Relative pi- Abundance in Likelihood Ratio Bins for `ALL' detector}
	\end{figure}
\end{frame}

\begin{frame}
	\frametitle{\insertsection}
	\framesubtitle{\insertsubsection}

	\begin{figure}
		\centering
		\includegraphics[width=\textwidth,height=0.7\textheight,keepaspectratio]{{{../res/charged 01/General Purpose Statistics: Relative K- Abundance in Likelihood Ratio Bins for ALL detector}}}
		\caption{Relative K- Abundance in Likelihood Ratio Bins for `ALL' detector}
	\end{figure}
\end{frame}

\subsection{`ALL' detector; positively charged}
\begin{frame}
	\frametitle{\insertsection}
	\framesubtitle{\insertsubsection}

	\begin{figure}
		\centering
		\includegraphics[width=\textwidth,height=0.7\textheight,keepaspectratio]{{{../res/charged 01/General Purpose Statistics: Relative pi+ Abundance in Likelihood Ratio Bins for ALL detector}}}
		\caption{Relative pi+ Abundance in Likelihood Ratio Bins for `ALL' detector}
	\end{figure}
\end{frame}

\begin{frame}
	\frametitle{\insertsection}
	\framesubtitle{\insertsubsection}

	\begin{figure}
		\centering
		\includegraphics[width=\textwidth,height=0.7\textheight,keepaspectratio]{{{../res/charged 01/General Purpose Statistics: Relative K+ Abundance in Likelihood Ratio Bins for ALL detector}}}
		\caption{Relative K+ Abundance in Likelihood Ratio Bins for `ALL' detector}
	\end{figure}
\end{frame}

\section{Appendix}
\subsection{Particle Abundance in MC9 Charged Decay}
\begin{frame}
	\frametitle{\insertsection}
	\framesubtitle{\insertsubsection}

	\centering
	\includegraphics[width=\textwidth,height=0.8\textheight,keepaspectratio]{{{../res/charged 01/General Purpose Statistics: True Particle Abundances in the K+-Data}}}
\end{frame}

\subsection{`SVD' detector}
\begin{frame}
	\frametitle{\insertsection}
	\framesubtitle{\insertsubsection}

	\centering
	\includegraphics[width=\textwidth,height=0.8\textheight,keepaspectratio]{{{../res/charged 01/General Purpose Statistics: Relative pi Abundance in Likelihood Ratio Bins for SVD detector}}}
\end{frame}

\subsection{`CDC' detector}
\begin{frame}
	\frametitle{\insertsection}
	\framesubtitle{\insertsubsection}

	\centering
	\includegraphics[width=\textwidth,height=0.8\textheight,keepaspectratio]{{{../res/charged 01/General Purpose Statistics: Relative pi Abundance in Likelihood Ratio Bins for CDC detector}}}
\end{frame}

\subsection{`TOP' detector}
\begin{frame}
	\frametitle{\insertsection}
	\framesubtitle{\insertsubsection}

	\centering
	\includegraphics[width=\textwidth,height=0.8\textheight,keepaspectratio]{{{../res/charged 01/General Purpose Statistics: Relative pi Abundance in Likelihood Ratio Bins for TOP detector}}}
\end{frame}

\subsection{`ARICH' detector}
\begin{frame}
	\frametitle{\insertsection}
	\framesubtitle{\insertsubsection}

	\centering
	\includegraphics[width=\textwidth,height=0.8\textheight,keepaspectratio]{{{../res/charged 01/General Purpose Statistics: Relative pi Abundance in Likelihood Ratio Bins for ARICH detector}}}
\end{frame}

\subsection{`ECL' detector}
\begin{frame}
	\frametitle{\insertsection}
	\framesubtitle{\insertsubsection}

	\centering
	\includegraphics[width=\textwidth,height=0.8\textheight,keepaspectratio]{{{../res/charged 01/General Purpose Statistics: Relative pi Abundance in Likelihood Ratio Bins for ECL detector}}}
\end{frame}

\subsection{`KLM' detector}
\begin{frame}
	\frametitle{\insertsection}
	\framesubtitle{\insertsubsection}

	\centering
	\includegraphics[width=\textwidth,height=0.8\textheight,keepaspectratio]{{{../res/charged 01/General Purpose Statistics: Relative pi Abundance in Likelihood Ratio Bins for KLM detector}}}
\end{frame}

\end{document}
