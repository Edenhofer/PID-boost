\documentclass{beamer}
	\usepackage[utf8]{inputenc}		% required for umlauts
	\usepackage[english]{babel}		% language
	%\usepackage[sfdefault]{roboto}	% enable sans serif font roboto
	%\usepackage{libertine}			% enable this on Windows to allow for microtype
	\usepackage[T1]{fontenc}		% required for output of umlauts in PDF

	\usepackage{mathtools}		% required for formulas

	\usepackage{caption}		% Customize caption aesthetics
	\usepackage{tcolorbox}		% fancy colored boxes
	\usepackage{xcolor}			% Highlighting
	\usepackage{soul}

	\usepackage{graphicx}		% required to insert images
	\usepackage{subcaption}		% enable sub-figure
	\usepackage[space]{grffile} % insert images baring a filename which contains spaces
	\usepackage{float}			% allow to forcefully set the location of an object

	\usepackage[tracking=true]{microtype} % required to change character spacing

	\usepackage[style=numeric,backend=biber]{biblatex}
	\usepackage{hyperref}		% insert clickable references

	\usepackage{datetime}		% flexible date specification
	\newcommand{\leadingzero}[1]{\ifnum#1<10 0\the#1\else\the#1\fi}
	\newcommand{\todayddmmyyyy}{\leadingzero{\day}.\leadingzero{\month}.\the\year}
	\newcommand{\mathcolorbox}[2]{\colorbox{#1}{$\displaystyle #2$}}

	\usepackage{geometry}
	\usepackage{scrextend}		% allow arbitrary indentation

	\usepackage{color}

	\setbeamercolor{title}{fg=orange}
	\setbeamertemplate{title}{
		\color{orange}
		\textbf{\inserttitle}
	}
	\setbeamercolor{tableofcontents}{fg=orange}
	\setbeamercolor{section in toc}{fg=black}
	\setbeamercolor{subsection in toc}{fg=black}
	\setbeamertemplate{frametitle}{
		%\vspace{0.5em}
		\color{orange}
		\begin{center}
			\textbf{\insertframetitle} \\
			{\small \insertframesubtitle}
		\end{center}
	}
	\setbeamertemplate{itemize item}{\color{black}$\bullet$}
	\setbeamertemplate{itemize subitem}{\color{black}$\circ$}
	\setbeamercolor{block title}{fg=black}
	\captionsetup{font=scriptsize,labelfont={bf,scriptsize}}
	\captionsetup[sub]{font=scriptsize,labelfont={bf,scriptsize}}

	\title{Bachelor Thesis on `Optimization~of~Particle~Identification'}
	\subtitle{Preliminary Studies}
	\author[Edenhofer]{Gordian Edenhofer}
	\institute[LMU]{
		Working Group of Prof.~Kuhr, Faculty of Physics \\
		University of Munich
	}
	\date{Physics Performance Meeting, 29. Mai 2018}
	\subject{Particle Physics}


\begin{document}

\section{Neyman Pearson}
\subsection{Anomalies}
\begin{frame}
	\frametitle{\insertsection}
	\framesubtitle{\insertsubsection}

	\begin{figure}
		\centering
		\includegraphics[width=\textwidth,height=0.7\textheight,keepaspectratio]{{{../res/charged 01/General Purpose Statistics: Relative p+ Abundance in Likelihood Ratio Bins for ALL detector}}}
		\caption{Relative $p^+$ Abundance in Likelihood Ratio Bins for the `ALL' detector.}
	\end{figure}
\end{frame}

\subsection{Anomalies by detector}
\begin{frame}
	\frametitle{\insertsection}
	\framesubtitle{\insertsubsection}

	\begin{figure}
		\centering
		\begin{subfigure}{0.32\textwidth}
			\centering
			\includegraphics[width=\textwidth,height=0.7\textheight,keepaspectratio]{{{../res/charged 01/General Purpose Statistics: Relative p+ Abundance in Likelihood Ratio Bins for SVD detector}}}
			\caption{`SVD'}
		\end{subfigure}
		\begin{subfigure}{0.32\textwidth}
			\centering
			\includegraphics[width=\textwidth,height=0.7\textheight,keepaspectratio]{{{../res/charged 01/General Purpose Statistics: Relative p+ Abundance in Likelihood Ratio Bins for CDC detector}}}
			\caption{`CDC'}
		\end{subfigure}
		\begin{subfigure}{0.32\textwidth}
			\centering
			\includegraphics[width=\textwidth,height=0.7\textheight,keepaspectratio]{{{../res/charged 01/General Purpose Statistics: Relative p+ Abundance in Likelihood Ratio Bins for TOP detector}}}
			\caption{`TOP'}
		\end{subfigure}

		\begin{subfigure}{0.32\textwidth}
			\centering
			\includegraphics[width=\textwidth,height=0.7\textheight,keepaspectratio]{{{../res/charged 01/General Purpose Statistics: Relative p+ Abundance in Likelihood Ratio Bins for ARICH detector}}}
			\caption{`ARICH'}
		\end{subfigure}
		\begin{subfigure}{0.32\textwidth}
			\centering
			\includegraphics[width=\textwidth,height=0.7\textheight,keepaspectratio]{{{../res/charged 01/General Purpose Statistics: Relative p+ Abundance in Likelihood Ratio Bins for ECL detector}}}
			\caption{`ECL'}
		\end{subfigure}
		\begin{subfigure}{0.32\textwidth}
			\centering
			\includegraphics[width=\textwidth,height=0.7\textheight,keepaspectratio]{{{../res/charged 01/General Purpose Statistics: Relative p+ Abundance in Likelihood Ratio Bins for KLM detector}}}
			\caption{`KLM'}
		\end{subfigure}

		\caption{Relative $p^+$ Abundance in Likelihood Ratio Bins for various detectors.}
	\end{figure}
\end{frame}

\subsection{Anomalies by charge}
\begin{frame}
	\frametitle{\insertsection}
	\framesubtitle{\insertsubsection}

	\begin{figure}
		\centering
		\begin{subfigure}{0.49\textwidth}
			\centering
			\includegraphics[width=\textwidth,height=0.7\textheight,keepaspectratio]{{{../res/charged 01/General Purpose Statistics: Relative p+ Abundance in Likelihood Ratio Bins for ALL detector}}}
			\caption{$p^+$}
		\end{subfigure}
		\begin{subfigure}{0.49\textwidth}
			\centering
			\includegraphics[width=\textwidth,height=0.7\textheight,keepaspectratio]{{{../res/charged 01/General Purpose Statistics: Relative barp- Abundance in Likelihood Ratio Bins for ALL detector}}}
			\caption{$\bar{p}^-$}
		\end{subfigure}

		\caption{Relative abundance in likelihood ratio bins for various detectors.}
	\end{figure}
\end{frame}

\subsection{Anomalies in bins}
\begin{frame}
	\frametitle{\insertsection}
	\framesubtitle{\insertsubsection}

	\begin{figure}
		\centering
		\includegraphics[width=\textwidth,height=0.6\textheight,keepaspectratio]{{{../res/charged 01/General Purpose Statistics: Relative p Abundance in Likelihood Ratio Bins for ALL detector for equal width pt bins}}}
		\caption{Relative $p$ Abundance in Likelihood Ratio Bins for the `ALL' detector using \textit{equal~width} $p_t$ bins.}
	\end{figure}
\end{frame}

\begin{frame}
	\frametitle{\insertsection}
	\framesubtitle{\insertsubsection}

	\begin{figure}
		\centering
		\includegraphics[width=\textwidth,height=0.6\textheight,keepaspectratio]{{{../res/charged 01/General Purpose Statistics: Relative p Abundance in Likelihood Ratio Bins for ALL detector for equal size pt bins}}}
		\caption{Relative $p$ Abundance in Likelihood Ratio Bins for the `ALL' detector using \textit{equal~size} $p_t$ bins.}
	\end{figure}
\end{frame}

\begin{frame}
	\frametitle{\insertsection}
	\framesubtitle{\insertsubsection}

	\begin{figure}
		\centering
		\includegraphics[width=\textwidth,height=0.6\textheight,keepaspectratio]{{{../res/charged 01/General Purpose Statistics: Relative p Abundance in Likelihood Ratio Bins for ALL detector for equal width cos(Theta) bins}}}
		\caption{Relative $p$ Abundance in Likelihood Ratio Bins for the `ALL' detector using \textit{equal~width} $cos(\Theta)$ bins.}
	\end{figure}
\end{frame}

\begin{frame}
	\frametitle{\insertsection}
	\framesubtitle{\insertsubsection}

	\begin{figure}
		\centering
		\includegraphics[width=\textwidth,height=0.6\textheight,keepaspectratio]{{{../res/charged 01/General Purpose Statistics: Relative p Abundance in Likelihood Ratio Bins for ALL detector for equal size cos(Theta) bins}}}
		\caption{Relative $p$ Abundance in Likelihood Ratio Bins for the `ALL' detector using \textit{equal~size} $cos(\Theta)$ bins.}
	\end{figure}
\end{frame}

\subsection{pidProbability scatter plot}
\begin{frame}
	\frametitle{\insertsection}
	\framesubtitle{\insertsubsection}

	\begin{figure}
		\centering
		\includegraphics[width=\textwidth,height=0.6\textheight,keepaspectratio]{{{../res/charged 01/General Purpose Statistics: p pidPropability multi-axes Histogram of pt, cosTheta}}}
		\caption{Multi-axes Histogram for the $p$-pidPropability as scatter plot depending on $p_t$ and $cos(\Theta)$.}
	\end{figure}
\end{frame}

\section{Confusion Matrix}
\subsection{Particle Identification}
\begin{frame}
	\frametitle{\insertsection}
	\framesubtitle{\insertsubsection}

	\begin{figure}
		\centering
		\includegraphics[width=\textwidth,height=0.7\textheight,keepaspectratio]{{{../res/charged 01/pidProbability Approach: Heatmap of epsilonPID Matrix for an exclusive Cut}}}
		\caption{Right stochastic confusion matrix for an exclusive `pidProbability' cut.}
	\end{figure}
\end{frame}

\section{Appendix}
\subsection{Particle Abundance in MC9 Charged Decay}
\begin{frame}
	\frametitle{\insertsection}
	\framesubtitle{\insertsubsection}

	\begin{figure}
		\centering
		\includegraphics[width=\textwidth,height=0.7\textheight,keepaspectratio]{{{../res/charged 01/General Purpose Statistics: True Particle Abundances in the K+-Data}}}
		\caption{True Particle Abundances in the simulated data.}
	\end{figure}
\end{frame}

\subsection{pidProbability scatter plot}
\begin{frame}
	\frametitle{\insertsection}
	\framesubtitle{\insertsubsection}

	\begin{figure}
		\centering
		\includegraphics[width=\textwidth,height=0.6\textheight,keepaspectratio]{{{../res/charged 01/General Purpose Statistics: pi pidPropability multi-axes Histogram of pt, cosTheta}}}
		\caption{Multi-axes Histogram for the $\pi$-pidPropability as scatter plot depending on $p_t$ and $cos(\Theta)$.}
	\end{figure}
\end{frame}

\end{document}
