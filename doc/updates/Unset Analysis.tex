\documentclass{beamer}
	\usepackage[utf8]{inputenc}		% required for umlauts
	\usepackage[english]{babel}		% language
	%\usepackage[sfdefault]{roboto}	% enable sans serif font roboto
	%\usepackage{libertine}			% enable this on Windows to allow for microtype
	\usepackage[T1]{fontenc}		% required for output of umlauts in PDF

	\usepackage{mathtools}		% required for formulas

	\usepackage{caption}		% Customize caption aesthetics
	\usepackage{tcolorbox}		% fancy colored boxes
	\usepackage{xcolor}			% Highlighting
	\usepackage{soul}

	\usepackage{listings}		% Insert programming code
	\usepackage{graphicx}		% required to insert images
	\usepackage{subcaption}		% enable sub-figure
	\usepackage[space]{grffile} % insert images baring a filename which contains spaces
	\usepackage{float}			% allow to forcefully set the location of an object

	\usepackage[tracking=true]{microtype} % required to change character spacing

	\usepackage[style=numeric,backend=biber]{biblatex}
	\usepackage{hyperref}		% insert clickable references

	\usepackage{datetime}		% flexible date specification
	\newcommand{\leadingzero}[1]{\ifnum#1<10 0\the#1\else\the#1\fi}
	\newcommand{\todayddmmyyyy}{\leadingzero{\day}.\leadingzero{\month}.\the\year}
	\newcommand{\mathcolorbox}[2]{\colorbox{#1}{$\displaystyle #2$}}

	\usepackage{geometry}
	\usepackage{scrextend}		% allow arbitrary indentation

	\usepackage{color}

	\setbeamercolor{title}{fg=orange}
	\setbeamertemplate{title}{
		\color{orange}
		\textbf{\inserttitle}
	}
	\setbeamercolor{tableofcontents}{fg=orange}
	\setbeamercolor{section in toc}{fg=black}
	\setbeamercolor{subsection in toc}{fg=black}
	\setbeamertemplate{frametitle}{
		%\vspace{0.5em}
		\color{orange}
		\begin{center}
			\textbf{\insertframetitle} \\
			{\small \insertframesubtitle}
		\end{center}
	}
	\setbeamertemplate{footline}[text line]{
		\parbox{\linewidth}{
			\color{gray}
			\vspace*{-1em}
			BA 2018
			\hfill
			Gordian (\href{mailto:gordian.edenhofer@gmail.com}{gordian.edenhofer@gmail.com})
			\hfill
			\insertpagenumber
		}
	}
	\setbeamertemplate{navigation symbols}{}
	\setbeamertemplate{itemize item}{\color{black}$\bullet$}
	\setbeamertemplate{itemize subitem}{\color{black}$\circ$}
	\setbeamercolor{block title}{fg=black}
	\captionsetup{font=scriptsize,labelfont={bf,scriptsize}}

	\lstset{basicstyle=\ttfamily,breaklines=true,showstringspaces=false,commentstyle=\color{red},keywordstyle=\color{blue}}

	\title{Bachelor Thesis in `Optimization~of~Particle~Identification'}
	\subtitle{Test on the likelihood ratio based on the Neyman Pearson lemma for different CDC detector settings}
	\author[Edenhofer]{\href{mailto:gordian.edenhofer@gmail.com}{Gordian Edenhofer}}
	\institute[LMU]{
		Working Group of Prof.~Dr.~Kuhr \\
		Faculty of Physics \\
		Excellence Cluster Universe
	}
	\date[BA Thesis 2018]{\today}
	\subject{Particle Physics}


\begin{document}

\section{Neyman Pearson}
\subsection{Intuition}
\begin{frame}
	\frametitle{\insertsection}
	\framesubtitle{\insertsubsection}

	\begin{columns}[T]
		\begin{column}{.45\textwidth}
			\vspace{2.5em}
			\begin{itemize}
				\item Increasingly strict $\mathcal{LR}$ criteria on x-axis
				\item Relative particle abundance for the given $\mathcal{LR}$ on y-axis
				\item \textbf{Ideally a monotonically increasing function}
			\end{itemize}
		\end{column}
		\begin{column}{.55\textwidth}
			\begin{figure}
				\centering
				\includegraphics[width=\textwidth,height=\textheight,keepaspectratio]{{{../res/Neyman-Pearson Visualization}}}
				\caption{Idealized example for relative $\pi$ Abundance over Likelihood Ratio.}
			\end{figure}
		\end{column}
	\end{columns}
\end{frame}

\section{Neyman Pearson}
\subsection{Anomalies}
\begin{frame}
	\frametitle{\insertsection}
	\framesubtitle{\insertsubsection}

	\begin{figure}
		\centering
		\includegraphics[width=\textwidth,height=0.68\textheight,keepaspectratio]{{{../res/charged 01/General Purpose Statistics: Relative p Abundance in Likelihood Ratio Bins for ALL detector}}}
		\caption{Relative $p$ Abundance in Likelihood Ratio Bins for the `ALL' detector.}
	\end{figure}
\end{frame}

\subsection{Applied Prior Selections}
\begin{frame}
	\frametitle{\insertsection}
	\framesubtitle{\insertsubsection}

	\begin{itemize}
		\item Disregard false identifications
		\item Ignore imprecise tracks
	\end{itemize}

	\begin{itemize}
		\item $0.05 < p_t < 5.29$
		\item $|z0| < 5$
		\item $|d0| < 2$
		\item $mcPDG \neq 0$
	\end{itemize}
\end{frame}

\subsection{Without \lstinline|BELLE2_CONDB_GLOBALTAG|}
\begin{frame}
	\frametitle{\insertsection}
	\framesubtitle{\insertsubsection}

	\begin{figure}
		\centering
		\includegraphics[width=\textwidth,height=0.6\textheight,keepaspectratio]{{{../res/unset/pidProbability Approach: Relative p Abundance in Likelihood Ratio Bins for CDC detector for equal size pt bins}}}
		\caption{Relative $p$ Abundance in Likelihood Ratio Bins for the CDC detector using \textit{equal~height} $p_t$ bins.}
	\end{figure}
\end{frame}

\subsection{With \lstinline|BELLE2_CONDB_GLOBALTAG|}
\begin{frame}
	\frametitle{\insertsection}
	\framesubtitle{\insertsubsection}

	\begin{figure}
		\centering
		\includegraphics[width=\textwidth,height=0.6\textheight,keepaspectratio]{{{../res/set/pidProbability Approach: Relative p Abundance in Likelihood Ratio Bins for CDC detector for equal size pt bins}}}
		\caption{Relative $p$ Abundance in Likelihood Ratio Bins for the CDC detector using \textit{equal~height} $p_t$ bins.}
	\end{figure}
\end{frame}

\section{Appendix}
\subsection{Event Generation using \lstinline|basf2|}
\begin{frame}[fragile]
	\frametitle{\insertsection}
	\framesubtitle{\insertsubsection}

	\begin{itemize}
		\item MDST files are generated in accordance with MC9 campaign
		\item Adapted version of the original script may be found at \href{https://github.com/Edenhofer/PID-boost/blob/master/generate_charged.py}{GitHub:Edenhofer/PID-boost}
	\end{itemize}

	\begin{lstlisting}[language=bash,xleftmargin=-0.16\textwidth,basicstyle=\ttfamily\tiny]
		#!/bin/bash

		source ~/setup_belle2.sh release-01-00-03

		unset BELLE2_CONDB_GLOBALTAG
		for n in {0..10}; do
			&>unset_$n.log \
			basf2 generate_charged.py -n 10000 -o unset_$n.mdst.root &
		done

		export BELLE2_CONDB_GLOBALTAG="Calibration_Offline_Development"
		for n in {0..10}; do
			&>set_$n.log \
			basf2 generate_charged.py -n 10000 -o set_$n.mdst.root &
		done
	\end{lstlisting}
\end{frame}

\end{document}
