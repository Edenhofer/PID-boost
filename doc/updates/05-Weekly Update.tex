\documentclass{beamer}
	\usepackage[utf8]{inputenc}		% required for umlauts
	\usepackage[english]{babel}		% required for umlauts
	\usepackage[sfdefault]{roboto}	% enable sans serif font roboto
	%\usepackage{libertine}			% enable this on Windows to allow for microtype
	\usepackage[T1]{fontenc}		% required for output of umlauts in PDF

	\usepackage{mathtools}		% required for formulas

	\usepackage{graphicx}		% required to insert images
	\usepackage{subcaption}		% enable sub-figure
	\usepackage[space]{grffile} % insert images baring a filename which contains spaces
	\usepackage{float}			% allow to forcefully set the location of an object

	\usepackage[tracking=true]{microtype} % required to change character spacing

	\usepackage{hyperref}		% insert clickable references

	\usepackage{datetime}		% flexible date specification
	\newcommand{\leadingzero}[1]{\ifnum#1<10 0\the#1\else\the#1\fi}
	\newcommand{\todayddmmyyyy}{\leadingzero{\day}.\leadingzero{\month}.\the\year}

	\usepackage{geometry}
	\usepackage{scrextend}		% allow arbitrary indentation

	\usepackage{color}

	\usetheme{default}
	\usecolortheme{beaver}

	\title{Fifth Weekly Update on `Optimization~of~Particle~Identification'}
	\subtitle{Neural Network using pidProbabilities, using main PCA Components, as Parameter Selector}
	\author{Gordian Edenhofer}
	\date{\today}


\begin{document}
\begin{frame}
	\frametitle{Git Log}

	\begin{itemize}
		\item Simple Neural Network using `pidProbabilities' as input
		\begin{itemize}
			\item{Train}
			\item{Validate}
			\item{Visualize}
		  \end{itemize}
		\item \dots using main PCA Components
		\begin{itemize}
			\item{Standardize data}
			\item{Pick $n$ principal components and transform data}
			\item{Repeat steps from simple}
		  \end{itemize}
	\end{itemize}
\end{frame}

\begin{frame}
	\frametitle{Neural Network}
	\framesubtitle{Model Architecture}

	\begin{columns}[T]
		\begin{column}{.5\textwidth}
			\vspace{4.5em}
			\begin{itemize}
				\item 5 $\times$ Dense
				\item 2 $\times$ Dropout
				\item Successive reduction in free Parameters
			\end{itemize}
		\end{column}
		\begin{column}{.5\textwidth}
			\begin{figure}
				\centering
				\includegraphics[width=\textwidth,height=0.6\textheight,keepaspectratio]{{{../res/Neural Network Design}}}
				\caption{Design of the Neural Network}
			\end{figure}
		\end{column}
	\end{columns}
\end{frame}

\begin{frame}
	\frametitle{Sample Decay --- Training}
	\framesubtitle{`pidProbability'}

	\begin{figure}
		\centering
		\begin{subfigure}{.49\textwidth}
			\centering
			\includegraphics[width=\linewidth]{{{../res/sample/Neural Network Model pidProbability: Training Accuracy with 20 epochs, with 64 batchsize, with 12 ncomponents}}}
			\caption{Accuracy}
		\end{subfigure}
		\begin{subfigure}{.49\textwidth}
			\centering
			\includegraphics[width=\linewidth]{{{../res/sample/Neural Network Model pidProbability: Training Loss with 20 epochs, with 64 batchsize, with 12 ncomponents}}}
			\caption{Loss}
		\end{subfigure}
		\caption{Training results of a neural network using `pidProbability'}
	\end{figure}
\end{frame}

\begin{frame}
	\frametitle{Sample Decay --- Validation Evolution}
	\framesubtitle{`pidProbability'}

	\begin{figure}
		\centering
		\begin{subfigure}{.49\textwidth}
			\centering
			\includegraphics[width=\linewidth]{{{../res/sample/Neural Network Model pidProbability: Validation Accuracy with 20 epochs, with 64 batchsize, with 12 ncomponents}}}
			\caption{Accuracy}
		\end{subfigure}
		\begin{subfigure}{.49\textwidth}
			\centering
			\includegraphics[width=\linewidth]{{{../res/sample/Neural Network Model pidProbability: Validation Loss with 20 epochs, with 64 batchsize, with 12 ncomponents}}}
			\caption{Loss}
		\end{subfigure}
		\caption{Validation results of a neural network using `pidProbability'}
	\end{figure}
\end{frame}


\begin{frame}
	\frametitle{Sample Decay --- Training}
	\framesubtitle{PCA}

	\begin{figure}
		\centering
		\begin{subfigure}{.49\textwidth}
			\centering
			\includegraphics[width=\linewidth]{{{../res/sample/Neural Network Model pca: Training Accuracy with 20 epochs, with 64 batchsize, with 12 ncomponents}}}
			\caption{Accuracy}
		\end{subfigure}
		\begin{subfigure}{.49\textwidth}
			\centering
			\includegraphics[width=\linewidth]{{{../res/sample/Neural Network Model pca: Training Loss with 20 epochs, with 64 batchsize, with 12 ncomponents}}}
			\caption{Loss}
		\end{subfigure}
		\caption{Training results of a neural network using PCA}
	\end{figure}
\end{frame}

\begin{frame}
	\frametitle{Sample Decay --- Validation Evolution}
	\framesubtitle{PCA}

	\begin{figure}
		\centering
		\begin{subfigure}{.49\textwidth}
			\centering
			\includegraphics[width=\linewidth]{{{../res/sample/Neural Network Model pca: Validation Accuracy with 20 epochs, with 64 batchsize, with 12 ncomponents}}}
			\caption{Accuracy}
		\end{subfigure}
		\begin{subfigure}{.49\textwidth}
			\centering
			\includegraphics[width=\linewidth]{{{../res/sample/Neural Network Model pca: Validation Loss with 20 epochs, with 64 batchsize, with 12 ncomponents}}}
			\caption{Loss}
		\end{subfigure}
		\caption{Validation results of a neural network using PCA}
	\end{figure}
\end{frame}

\begin{frame}
	\frametitle{Sample Decay --- Method comparisons}
	\framesubtitle{$\epsilon_{PID}$-matrix Visualization}

	\begin{figure}
		\centering
		\includegraphics[width=\textwidth,height=0.8\textheight,keepaspectratio]{{{../res/sample/Diff Heatmap: Heatmap of epsilonPID Matrix for an exclusive Cut via flat Bayes, via NN}}}
		\caption{Heatmap of $\epsilon_{PID}$ Matrix for an exclusive Cut via flat Bayes and via a neural network using PCA}
	\end{figure}
\end{frame}

\begin{frame}
	\frametitle{Appendix --- Charged Decay --- Training}
	\framesubtitle{PCA}

	\begin{figure}
		\centering
		\begin{subfigure}{.49\textwidth}
			\centering
			\includegraphics[width=\linewidth]{{{../res/charged 01/Neural Network Model pca: Training Accuracy with 20 epochs, with 64 batchsize, with 12 ncomponents}}}
			\caption{Accuracy}
		\end{subfigure}
		\begin{subfigure}{.49\textwidth}
			\centering
			\includegraphics[width=\linewidth]{{{../res/charged 01/Neural Network Model pca: Training Loss with 20 epochs, with 64 batchsize, with 12 ncomponents}}}
			\caption{Loss}
		\end{subfigure}
		\caption{Training results of a neural network using PCA}
	\end{figure}
\end{frame}

\begin{frame}
	\frametitle{Appendix --- Charged Decay --- Validation Evolution}
	\framesubtitle{PCA}

	\begin{figure}
		\centering
		\begin{subfigure}{.49\textwidth}
			\centering
			\includegraphics[width=\linewidth]{{{../res/charged 01/Neural Network Model pca: Validation Accuracy with 20 epochs, with 64 batchsize, with 12 ncomponents}}}
			\caption{Accuracy}
		\end{subfigure}
		\begin{subfigure}{.49\textwidth}
			\centering
			\includegraphics[width=\linewidth]{{{../res/charged 01/Neural Network Model pca: Validation Loss with 20 epochs, with 64 batchsize, with 12 ncomponents}}}
			\caption{Loss}
		\end{subfigure}
		\caption{Validation results of a neural network using PCA}
	\end{figure}
\end{frame}

\begin{frame}
	\frametitle{Appendix --- Charged Decay --- Method comparisons}
	\framesubtitle{$\epsilon_{PID}$-matrix Visualization}

	\begin{figure}
		\centering
		\includegraphics[width=\textwidth,height=0.8\textheight,keepaspectratio]{{{../res/charged 01/Diff Heatmap: Heatmap of epsilonPID Matrix for an exclusive Cut via flat Bayes, via NN}}}
		\caption{Heatmap of $\epsilon_{PID}$ Matrix for an exclusive Cut via flat Bayes and via a neural network using PCA}
	\end{figure}
\end{frame}

\end{document}
